% =============================================================================
% LaTeX Workshop Slide Deck — Northeastern University Wireless Club
% =============================================================================
\documentclass[x11names]{beamer} % Global option [x11names] avoids xcolor clash
\usetheme{Madrid}
\usecolortheme{beaver}

% =============================================================================
% Customization: Set a darker gray-ish color for the "section in head/foot"
% (This is the bottom-right footer column used for displaying the version (Git commit hash)
% and the frame numbering.)
% =============================================================================
\setbeamercolor{section in head/foot}{bg=gray!60, fg=black}


% =============================================================================
% Packages
% =============================================================================
\usepackage{graphicx}   % For including images (e.g., QR codes)
\usepackage{hyperref}   % For clickable hyperlinks
\usepackage{xcolor}     % For defining custom colors (e.g., the urlcolor)
\usepackage{iexec}      % Run command line (used for Git hash in versioning)
\usepackage{tcolorbox}  % For colored boxes
\usepackage{qrcode}     % For generating QR codes
\usepackage[utf8]{inputenc}
\usepackage{mdframed}   % For framed environments
\usepackage{minted}     % For syntax-highlighted code blocks
\usepackage{tikz}       % For drawing (used in mdframed titles)

% =============================================================================
% Versioning (Git commit hash for version info in footer)
% =============================================================================
\newcommand{\gitAbbrevHash}{\iexec{git rev-parse --short HEAD}}

% =============================================================================
% Custom Environments (Defined for future use, currently not used)
% =============================================================================
\newenvironment{question}[1][]{
  \ifstrempty{#1}{}{
    \mdfsetup{
      frametitle={
        \tikz[baseline=(current bounding box.east),outer sep=0pt]
        \node[anchor=east,rectangle,fill=gray!30]{#1};
      }}
  }
  \mdfsetup{
    innertopmargin=10pt,
    linecolor=gray!30,
    linewidth=2pt,
    topline=true,
    frametitleaboveskip=\dimexpr - \ht\strutbox\relax
  }
  \begin{mdframed}
}{\end{mdframed}}

% =============================================================================
% Color + Hyperlink Configuration
% =============================================================================
\definecolor{nuRed}{HTML}{C8102E}  % Northeastern primary red

\hypersetup{
  colorlinks=true,
  linkcolor=blue,
  urlcolor=nuRed
}

% =============================================================================
% Footer Configuration
% =============================================================================
\setbeamertemplate{footline}{%
  \leavevmode%
  \hbox{
    \begin{beamercolorbox}[wd=.35\paperwidth,ht=2.25ex,dp=1ex,leftskip=.3cm]{author in head/foot}%
      Northeastern University Wireless Club%
    \end{beamercolorbox}%
    \begin{beamercolorbox}[wd=.20\paperwidth,ht=2.25ex,dp=1ex,center]{title in head/foot}%
      \LaTeX\ Workshop%
    \end{beamercolorbox}%
    \begin{beamercolorbox}[wd=.15\paperwidth,ht=2.25ex,dp=1ex,center]{date in head/foot}%
      Spring 2025%
    \end{beamercolorbox}%
    \begin{beamercolorbox}[wd=.30\paperwidth,ht=2.25ex,dp=1ex,leftskip=.3cm, rightskip=.3cm plus1fil]{section in head/foot}%
      Version: \gitAbbrevHash{} \hspace{1em} \insertframenumber/\inserttotalframenumber%
    \end{beamercolorbox}}%
  \vskip0pt%
}

% =============================================================================
% Slide: Title Page
% =============================================================================
\title{\LaTeX\ Workshop}
\author{Northeastern University Wireless Club - \large\texttt{W1KBN}}
\date{March 31, 2025}

\begin{document}

% =============================================================================
% Slide: Sign-In with QR Code
% =============================================================================
\begin{frame}
  \titlepage
  \vspace{-1cm}
  \begin{center}
    \textbf{Sign in Here:} \\
    \qrcode{https://l.w1kbn.org/signin} \\
    {\small \url{https://l.w1kbn.org/signin}}
  \end{center}
\end{frame}

% =============================================================================
% Slide: About + QR Codes
% =============================================================================
\begin{frame}
  \begin{tcolorbox}[colframe=black, colback=blue!10, title=About, center title]
    \begin{itemize}
      \item Regular Meetings: Thursdays, 7 PM @ Hayden Hall, Room 503
      \item Workshops: Mondays, 7 PM @ Forsyth Hall, Room 201
    \end{itemize}        
  \end{tcolorbox}

  \begin{tcolorbox}[colframe=black, colback=blue!10, title=QR Codes \& Links, center title]
    \begin{minipage}{0.32\textwidth}
      \centering
      \textbf{Sign in} \\
      \qrcode{https://l.w1kbn.org/signin} \\
      \tiny \url{https://l.w1kbn.org/signin}
    \end{minipage}
    \begin{minipage}{0.32\textwidth}
      \centering
      \textbf{Slack} \\
      \qrcode{https://neuwireless.slack.com/join/signup} \\
      \tiny \url{https://neuwireless.slack.com/join/signup}
    \end{minipage}
    \begin{minipage}{0.32\textwidth}
      \centering
      \textbf{Mailing List} \\
      \qrcode{http://eepurl.com/gduCIr} \\
      \tiny \url{http://eepurl.com/gduCIr}
    \end{minipage}
    \begin{itemize}
      \item Website: \url{https://nuwireless.org/}
    \end{itemize}
  \end{tcolorbox}
\end{frame}

% =============================================================================
% Slide: Workshop Goals and Structure
% =============================================================================
\begin{frame}{Workshop Goals and Structure}
    \textbf{Goals:}
    \begin{itemize}
        \item Understand what \LaTeX{} is and why it matters for students in STEM.
        \item Learn how to start using \LaTeX{} via Overleaf and other tools.
        \item Walk away ready to create your first \LaTeX\ document — or improve your existing ones.
    \end{itemize}
    \textbf{Structure:}
    \begin{enumerate}
        \item What is \LaTeX{}? \textit{(The Why)}
        \item How to Use \LaTeX{} \textit{(The How)}
        \item Hands-On Examples in Overleaf
        \item Helpful Resources \& Tips
    \end{enumerate}
\end{frame}

% =============================================================================
% Slide: What is LaTeX?
% =============================================================================
\begin{frame}{What is \LaTeX{}} 
\begin{itemize}
    \item A \textit{high-quality} typesetting system (used for technical and scientific documentation).
    \item Based on plain text and compiled into PDFs.
    \item Especially powerful for math, citations, figures, complex formatting, and \textit{code}.
    \item Common in academia, research, and publishing — and many professors encourage it!
\end{itemize}
\end{frame}

% =============================================================================
% Slide: TeX vs LaTeX
% =============================================================================
\begin{frame}{\TeX\ vs \LaTeX{}}
    \begin{itemize}
        \item \textbf{\href{https://tug.org/}{\TeX}}: A typesetting engine created by Donald Knuth in the late 1970s.
        \begin{itemize}
            \item Low-level, extremely powerful, but hard to use directly.
            \item Similar to how Git is the underlying version control tool.
        \end{itemize}
        
        \item \textbf{\href{https://www.latex-project.org/}{\LaTeX}}: A markup language and set of macros built on top of \TeX\ (by Leslie Lamport).
        \begin{itemize}
            \item Makes \TeX\ accessible, structured, and easier to use for most people.
            \item Similar to how GitHub provides a user-friendly interface to Git.
        \end{itemize}
    \end{itemize}
    \end{frame}

% =============================================================================
% Slide: Why LaTeX vs Microsoft Word
% =============================================================================
\begin{frame}{Why Use \LaTeX{} over Microsoft Word?}
    \begin{columns}[t] % <-- Align columns at the top
        \begin{column}{0.48\textwidth}
        \textbf{Microsoft Word (WYSIWYG\\[0.1em] \scriptsize{What You See Is What You Get})}
        \begin{itemize}
            \item Click-and-drag formatting
            \item Can be inconsistent
            \item Formatting breaks easily
            \item Great for casual documents
        \end{itemize}
        \end{column}
        \begin{column}{0.48\textwidth}
        \textbf{\LaTeX{} (WYSIWYT\\[0.1em] \scriptsize{What You See Is What You Type})}
        \begin{itemize}
            \item Precise, consistent output
            \item Separates content from formatting
            \item Portable, version-controlled
            \item Professional, especially for STEM
        \end{itemize}
        \end{column}
    \end{columns}
    \end{frame}    

% =============================================================================
% Slide: Real-World Applications
% =============================================================================
\begin{frame}{Real-World Applications of \LaTeX}
\begin{itemize}
    \item Resumes and CVs
    \item Lab reports and technical documentation
    \item Research papers, journal articles
    \item Thesis/dissertation formatting
    \item Slide decks, like this one (\LaTeX\ Beamer)!
\end{itemize}
\end{frame}


% =============================================================================
% Slide: How to Use LaTeX
% =============================================================================
\begin{frame}{How to Use \LaTeX}
    \begin{itemize}
        \item \textbf{Online:} Overleaf — no setup, collaborative, and free Pro w/ NU email! \\
        \textit{We'll be focusing on Overleaf today.}
        \item \textbf{Distributions:}
        \begin{itemize}
            \item macOS: \texttt{MacTeX}
            \item Linux: \texttt{TeX Live}
            \item Windows: \texttt{MiKTeX}, \texttt{TeX Live}
        \end{itemize}
        \item \textbf{Editors:} VS Code (\href{https://marketplace.visualstudio.com/items?itemName=James-Yu.latex-workshop}{\texttt{latex-workshop}}), Vim (\texttt{vimtex}), Emacs (\texttt{AUCTeX})
    \end{itemize}
\end{frame}    

% =============================================================================
% Slide: Why Overleaf is Perfect for Beginners
% =============================================================================
\begin{frame}{Why Overleaf is Perfect for Beginners}
    \begin{itemize}
        \item Handles all the heavy lifting “under the hood”; Overleaf takes care of compiling, packages, and setup so you can just focus on writing.
        \item Cloud-based, no install required.
        \item Real-time preview and auto-compile.
        \item Supports collaboration — great for group projects.
        \item Version control built-in.
        \item Connect your Northeastern email for \textbf{free Overleaf Pro}. \url{https://www.overleaf.com/edu/northeastern}
    \end{itemize}
\end{frame}    

% =============================================================================
% Slide: Basic Document Structure
% =============================================================================
\begin{frame}[fragile]{Basic Document Structure}
    \begin{columns}[t] % <-- Align columns at the top
        \begin{column}{0.55\textwidth}
        \textbf{Source (\LaTeX\ Code):}
        \begin{minted}[fontsize=\scriptsize]{latex}
\documentclass{article}
\begin{document}
Hello, world!
\end{document}
        \end{minted}
        \end{column}

        \begin{column}{0.45\textwidth}
        \textbf{Output:}
        \vspace{1em}
        \fbox{\parbox{.9\linewidth}{Hello, world!}}
        \end{column}
    \end{columns}
\end{frame}

    
% =============================================================================
% Slide: Common LaTeX Environments
% =============================================================================
\begin{frame}[fragile]{Common LaTeX Environments}
    \begin{columns}[t] % <-- Align columns at the top
        \begin{column}{0.55\textwidth}
        \textbf{Source (\LaTeX\ Code):}
        \begin{minted}[fontsize=\scriptsize]{latex}
\begin{itemize}
    \item Bullet points
\end{itemize}

\begin{enumerate}
    \item Ordered list
\end{enumerate}

\begin{equation}
    E = mc^2
\end{equation}
        \end{minted}
        \end{column}

        \begin{column}{0.45\textwidth}
        \textbf{Output:}
        \vspace{0.5em}
        \fbox{
          \parbox{.9\linewidth}{
            \begin{itemize}
              \item Bullet points
            \end{itemize}
            \begin{enumerate}
              \item Ordered list
            \end{enumerate}
            \vspace{1em}
            \begin{equation}
                E=mc^2
            \end{equation}
          }
        }
        \end{column}
    \end{columns}
    \vspace{1cm}
    ... and more here: \url{https://www.overleaf.com/learn/latex/Environments}
\end{frame}

% =============================================================================
% Slide: Live Example / Demo
% =============================================================================
\begin{frame}{Let's Head to Overleaf!}
\centering
\Huge{\textbf{Time for a live demo!}}\\[1em]
\Large{\url{https://www.overleaf.com}}
\end{frame}

% =============================================================================
% Slide: Tools and Resources
% =============================================================================
\begin{frame}{Tools and Resources}
    \begin{itemize}
        \item \textbf{Overleaf Documentation}: \url{https://www.overleaf.com/learn} — excellent for learning both Overleaf and \LaTeX\ fundamentals, even if you compile locally.
        \item \textbf{\LaTeX\ Templates}: \url{https://www.overleaf.com/latex/templates} or \url{https://www.latextemplates.com}
        \item \textbf{Cheat Sheets}: \url{https://quickref.me/latex.html} or \url{https://wch.github.io/latexsheet/}
        \item \textbf{Detexify}: \url{http://detexify.kirelabs.org} — draw a symbol to get its corresponding \LaTeX\ command.
        \item \texttt{amsmath}, \texttt{siunitx}, \texttt{minted}, \texttt{circuitikz} — essential packages for math, units, syntax highlighting, and circuit diagrams.
        \item \textbf{CTAN (Comprehensive TeX Archive Network)}: \url{https://ctan.org} — the main repository for all \LaTeX\ packages, especially useful for local development.
    \end{itemize}
\end{frame}

% =============================================================================
% Slide: Final Thoughts
% =============================================================================
\begin{frame}{Final Thoughts}
    \begin{itemize}
        \item \LaTeX\ may seem intimidating at first, but it's a powerful tool for creating high-quality, consistent documents.
        \item Overleaf makes it accessible to everyone — but as your skills grow, we encourage you to explore developing locally to better understand what's happening under the hood.
        \item Embrace the learning curve — your future self will thank you!
    \end{itemize}
    \vspace{1em}
    This entire slide deck was compiled locally using \LaTeX\ (Beamer)!
\end{frame}  

% =============================================================================
% Slide: Contact Information
% =============================================================================
\begin{frame}
    \frametitle{Contact Us}
    \begin{itemize}
        \item Questions? Feel free to reach out!
        \item Workshop Team Emails: \\
        {\href{mailto:elarbi.m@northeastern.edu}{elarbi.m}, 
        \href{mailto:aviedov.v@northeastern.edu}{aviedov.v}, 
        \href{mailto:heaney.ma@northeastern.edu}{heaney.ma}}[at]northeastern[d0t]edu
        \item General Workshop Email: \href{mailto:workshops@nuwireless.org}{workshops}[at]nuwireless[d0t]org
        \item Website: \url{https://nuwireless.org/}
        \item Location: Hayden Hall, Room 503
    \end{itemize}
    \vspace{1cm}
    \begin{flushright}
        \footnotesize{© 2025 Northeastern Wireless Club} \\
        \footnotesize{Design: \href{https://melarbi.com}{Muhammad Elarbi}, based on \LaTeX\ Beamer}\\
    \end{flushright}
\end{frame}

\end{document}
